Dans le cadre de la formation mastère spécialisé de l'esiea, il nous était demandé d'écrire un programme capable de lire des fichiers pcap. 
Le programme doit respecter un cahier des charges qui implique d'utiliser le langage de programmation Java qui est orienté objet. Il ne doit pas contenir d'interface graphique, simplement une interface en console. L'utilisation d'une API, autre que l'API standard Java est interdite.
Les différents protocoles devront être detecté par le programme, une option donné au programme permet de choisir le protocole que l'on souhaite afficher.
\begin{itemize}
    \item Ethernet;
    \item ARP;
    \item IP;
    \item TCP;
    \item UDP;
    \item ICMP;
    \item DHCP;
    \item HTTP;
    \item DNS.
\end{itemize}
Dans ce projet, je n'ai pas implémenter tout les protocoles. Il me manque les protocoles de couche 5 à 7 (DHCP, HTTP, DNS)
