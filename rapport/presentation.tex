\chapter{Présentation du projet}


\section{Introduction}

.Dans le cadre de la formation mastère spécialisé de l'esiea, il nous était demandé d'écrire un programme capable de lire des fichiers pcap. 
Le programme doit respecter un cahier des charges qui implique d'utiliser le langage de programmation Java qui est orienté objet. Il ne doit pas contenir d'interface graphique, simplement une interface en console. L'utilisation d'une API, autre que l'API standard Java est interdite.
Les différents protocoles devront être detecté par le programme, une option donné au programme permet de choisir le protocole que l'on souhaite afficher.
\begin{itemize}
    \item Ethernet;
    \item ARP;
    \item IP;
    \item TCP;
    \item UDP;
    \item ICMP;
    \item DHCP;
    \item HTTP;
    \item DNS.
\end{itemize}
Dans ce projet, je n'ai pas implémenter tout les protocoles. Il me manque les protocoles de couche 5 à 7 (DHCP, HTTP, DNS)



\section{Présentation du fichier Pcap}

La première chose a effectué lors de l'ouverture du fichier pcap, est l'extraction du "Global Header" du fichier.
Il contient plusieurs informations essentielles à la compréhension du fichier, notamment l'encodage des informations binaire.
Si Les informations sont codés selon la norme "Big-Endian" dans ce cas, on accepte d'ouvrir le fichier, sinon on le rejette.
Pour la norme Little-Endian il faut inverser chaque octet.

La figure ci-dessous montre comment est composé le fichier pcap.
\begin{figure}[!h]
    \begin{center}
\includegraphics[width=15cm]{./globalHeader.png}
    \end{center}
\end{figure}

Dans ce fichier, chaque paquet est composé de deux choses:
\begin{itemize}
    \item Un Header Pcap;
    \item Un Packet Data.
\end{itemize}       
Le header donne les informations de base du paquet.
Les données correspondent au paquet réseau. 
Pour commencer, j'ai créé un objet pour rassembler les informations contenu dans le pcap header, et un second objet qui correspond au donnée.


\section{Les différents objets du projet}

Les réseaux sont construit autour du modèle OSI, lui-même divisé en 7 couches.
Pour ce projet, j'ai créé un objet par couche. Les informations contenu dans l'objet "Header Pcap" contiennent les données sur la première couche de ce modèle.

        

%%inclusion d'une mage dans le document
%\begin{figure}[!h]
%\begin{center}
%%taille de l'image en largeur
%%remplacer "width" par "height" pour régler la hauteur
%\includegraphics[width=15cm]{presentation/schema}
%\end{center}
%%légende de l'image
%\caption{Schéma descriptif}
%\end{figure}
%
%%Contenu de la note précédemment marquée avec \footnotemark
%\footnotetext{Note bas de page "intro"}
%
%Bla
%%retour à la ligne (alinea)
%
%Bla\\
%%saut de paragraphe
%
%Bla
%
%\newpage
%
%\section{Problématique soulevée}
%
%Bla
%
%\begin{center}
%Problématique du sujet
%\end{center}
%
%\section{Hypothèse de solution}
%
%%Quoi :
%Bla\\
%
%Voici une liste :
%\begin{itemize}
%\item item 1;
%\item item 2;
%\item item 3;
%\item item 4.
%\end{itemize}
%
%Bla\\
%
%%Comment :
%Bla
%
%Bla\footnotemark\\
%
%%Detail :
%Bla(cf. ref. \cite{cite6}).
%%citation référencé dans le document "bibliographie.bib" inclus à la fin du document
%
%\footnotetext{Note bas de page "bla"}
